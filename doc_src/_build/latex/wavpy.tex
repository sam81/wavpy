%% Generated by Sphinx.
\def\sphinxdocclass{report}
\documentclass[letterpaper,10pt,english]{sphinxmanual}
\ifdefined\pdfpxdimen
   \let\sphinxpxdimen\pdfpxdimen\else\newdimen\sphinxpxdimen
\fi \sphinxpxdimen=.75bp\relax
\ifdefined\pdfimageresolution
    \pdfimageresolution= \numexpr \dimexpr1in\relax/\sphinxpxdimen\relax
\fi
%% let collapsible pdf bookmarks panel have high depth per default
\PassOptionsToPackage{bookmarksdepth=5}{hyperref}


\PassOptionsToPackage{warn}{textcomp}
\usepackage[utf8]{inputenc}
\ifdefined\DeclareUnicodeCharacter
% support both utf8 and utf8x syntaxes
  \ifdefined\DeclareUnicodeCharacterAsOptional
    \def\sphinxDUC#1{\DeclareUnicodeCharacter{"#1}}
  \else
    \let\sphinxDUC\DeclareUnicodeCharacter
  \fi
  \sphinxDUC{00A0}{\nobreakspace}
  \sphinxDUC{2500}{\sphinxunichar{2500}}
  \sphinxDUC{2502}{\sphinxunichar{2502}}
  \sphinxDUC{2514}{\sphinxunichar{2514}}
  \sphinxDUC{251C}{\sphinxunichar{251C}}
  \sphinxDUC{2572}{\textbackslash}
\fi
\usepackage{cmap}
\usepackage[T1]{fontenc}
\usepackage{amsmath,amssymb,amstext}
\usepackage{babel}



\usepackage{tgtermes}
\usepackage{tgheros}
\renewcommand{\ttdefault}{txtt}



\usepackage[Bjarne]{fncychap}
\usepackage{sphinx}

\fvset{fontsize=auto}
\usepackage{geometry}


% Include hyperref last.
\usepackage{hyperref}
% Fix anchor placement for figures with captions.
\usepackage{hypcap}% it must be loaded after hyperref.
% Set up styles of URL: it should be placed after hyperref.
\urlstyle{same}

\addto\captionsenglish{\renewcommand{\contentsname}{Contents:}}

\usepackage{sphinxmessages}
\setcounter{tocdepth}{1}



\title{wavpy}
\date{May 08, 2023}
\release{0.2.7}
\author{Samuele Carcagno}
\newcommand{\sphinxlogo}{\vbox{}}
\renewcommand{\releasename}{Release}
\makeindex
\begin{document}

\ifdefined\shorthandoff
  \ifnum\catcode`\=\string=\active\shorthandoff{=}\fi
  \ifnum\catcode`\"=\active\shorthandoff{"}\fi
\fi

\pagestyle{empty}
\sphinxmaketitle
\pagestyle{plain}
\sphinxtableofcontents
\pagestyle{normal}
\phantomsection\label{\detokenize{index::doc}}


\sphinxstepscope


\chapter{Introduction}
\label{\detokenize{intro:introduction}}\label{\detokenize{intro::doc}}\begin{quote}\begin{description}
\sphinxlineitem{Author}
\sphinxAtStartPar
Samuele Carcagno

\end{description}\end{quote}

\sphinxAtStartPar
Module for reading and writing WAV files using MATLAB\sphinxhyphen{}style wavread and wavwrite functions. It is a simple but convenient wrapper to the scipy.io.wavfile module.


\chapter{\sphinxstyleliteralintitle{\sphinxupquote{wavpy}} \textendash{} Module to read and write WAVs}
\label{\detokenize{intro:module-wavpy}}\label{\detokenize{intro:wavpy-module-to-read-and-write-wavs}}\label{\detokenize{intro:module-label}}\index{module@\spxentry{module}!wavpy@\spxentry{wavpy}}\index{wavpy@\spxentry{wavpy}!module@\spxentry{module}}
\sphinxAtStartPar
Module for reading and writing WAV files. It is a simple but convenient wrapper to the wave module and the scipy.io.wavfile module.
\index{sound() (in module wavpy)@\spxentry{sound()}\spxextra{in module wavpy}}

\begin{fulllineitems}
\phantomsection\label{\detokenize{intro:wavpy.sound}}
\pysigstartsignatures
\pysiglinewithargsret{\sphinxcode{\sphinxupquote{wavpy.}}\sphinxbfcode{\sphinxupquote{sound}}}{\emph{\DUrole{n}{snd}}, \emph{\DUrole{n}{fs}\DUrole{o}{=}\DUrole{default_value}{48000}}, \emph{\DUrole{n}{nbits}\DUrole{o}{=}\DUrole{default_value}{32}}}{}
\pysigstopsignatures
\sphinxAtStartPar
Play out a numpy array through the soundcard.


\section{Parameters}
\label{\detokenize{intro:parameters}}\begin{description}
\sphinxlineitem{snd}{[}array of floats{]}
\sphinxAtStartPar
The sound to be played.

\sphinxlineitem{fs}{[}int{]}
\sphinxAtStartPar
Sampling frequency of the sound.

\sphinxlineitem{nbits}{[}int{]}
\sphinxAtStartPar
Desired bit depth.

\end{description}


\section{Examples}
\label{\detokenize{intro:examples}}
\begin{sphinxVerbatim}[commandchars=\\\{\}]
\PYG{g+gp}{\PYGZgt{}\PYGZgt{}\PYGZgt{} }\PYG{n}{sound}\PYG{p}{(}\PYG{n}{snd}\PYG{p}{,} \PYG{n}{fs}\PYG{o}{=}\PYG{l+m+mi}{48000}\PYG{p}{,} \PYG{n}{nbits}\PYG{o}{=}\PYG{l+m+mi}{32}\PYG{p}{)}
\end{sphinxVerbatim}

\end{fulllineitems}

\index{wavread() (in module wavpy)@\spxentry{wavread()}\spxextra{in module wavpy}}

\begin{fulllineitems}
\phantomsection\label{\detokenize{intro:wavpy.wavread}}
\pysigstartsignatures
\pysiglinewithargsret{\sphinxcode{\sphinxupquote{wavpy.}}\sphinxbfcode{\sphinxupquote{wavread}}}{\emph{\DUrole{n}{fName}}, \emph{\DUrole{n}{scale}\DUrole{o}{=}\DUrole{default_value}{True}}}{}
\pysigstopsignatures
\sphinxAtStartPar
Read a WAV file into a numpy array.


\section{Parameters}
\label{\detokenize{intro:id1}}\begin{description}
\sphinxlineitem{fName}{[}string{]}
\sphinxAtStartPar
Name of the WAV file to read

\sphinxlineitem{scale}{[}boolean{]}
\sphinxAtStartPar
Option valid only for the PCM wave format. If \sphinxtitleref{True} the
data will be returned as floaring point values ranging
between \sphinxhyphen{}1 and 1. If \sphinxtitleref{False} the data will be returned
as the closest numpy integer type to the WAV bit depth,
with values randing within the bit depth range.

\end{description}


\section{Returns}
\label{\detokenize{intro:returns}}
\sphinxAtStartPar
snd : numpy array with the sound.

\sphinxAtStartPar
fs : sampling frequency.

\sphinxAtStartPar
nbits : bit depth.


\section{Examples}
\label{\detokenize{intro:id2}}
\begin{sphinxVerbatim}[commandchars=\\\{\}]
\PYG{g+gp}{\PYGZgt{}\PYGZgt{}\PYGZgt{} }\PYG{n}{snd}\PYG{p}{,} \PYG{n}{fs}\PYG{p}{,} \PYG{n}{nbits} \PYG{o}{=} \PYG{n}{wavread}\PYG{p}{(}\PYG{l+s+s2}{\PYGZdq{}}\PYG{l+s+s2}{file.wav}\PYG{l+s+s2}{\PYGZdq{}}\PYG{p}{)}
\end{sphinxVerbatim}

\end{fulllineitems}

\index{wavwrite() (in module wavpy)@\spxentry{wavwrite()}\spxextra{in module wavpy}}

\begin{fulllineitems}
\phantomsection\label{\detokenize{intro:wavpy.wavwrite}}
\pysigstartsignatures
\pysiglinewithargsret{\sphinxcode{\sphinxupquote{wavpy.}}\sphinxbfcode{\sphinxupquote{wavwrite}}}{\emph{\DUrole{n}{data}}, \emph{\DUrole{n}{fs}}, \emph{\DUrole{n}{nbits}}, \emph{\DUrole{n}{fName}}, \emph{\DUrole{n}{wave\_format}\DUrole{o}{=}\DUrole{default_value}{\textquotesingle{}PCM\textquotesingle{}}}, \emph{\DUrole{n}{scale}\DUrole{o}{=}\DUrole{default_value}{True}}}{}
\pysigstopsignatures
\sphinxAtStartPar
Write a numpy array as a WAV file.


\section{Parameters}
\label{\detokenize{intro:id3}}\begin{description}
\sphinxlineitem{data}{[}array of floats{]}
\sphinxAtStartPar
The data to be written to the WAV file.

\sphinxlineitem{fs}{[}int{]}
\sphinxAtStartPar
Sampling frequency of the sound.

\sphinxlineitem{nbits}{[}int{]}
\sphinxAtStartPar
Bit depth of the WAV file (currently only values of 16 and 32 are supported)

\sphinxlineitem{fName}{[}string{]}
\sphinxAtStartPar
Name of the WAV file.

\sphinxlineitem{scale}{[}boolean{]}
\sphinxAtStartPar
Option valid only for the PCM wave format. If the data are floating point
values ranging from \sphinxhyphen{}1 to 1 and scale is set to \sphinxtitleref{True} they will be converted
to the range of the appropriate integer type (according to the chosen bit depth).
If scale is set to \sphinxtitleref{False} it is assumed that the values are already in the range
of the appropriate integer type (e.g. between \sphinxhyphen{}2**15 and 2**15\sphinxhyphen{}1 for 16 bits).
Note that if \sphinxtitleref{wave\_format} is set to \sphinxtitleref{IEEE\_FLOAT} the data are never scaled.

\end{description}


\section{Examples}
\label{\detokenize{intro:id4}}
\begin{sphinxVerbatim}[commandchars=\\\{\}]
\PYG{g+gp}{\PYGZgt{}\PYGZgt{}\PYGZgt{} }\PYG{n}{wavwrite}\PYG{p}{(}\PYG{n}{data}\PYG{p}{,} \PYG{l+m+mi}{48000}\PYG{p}{,} \PYG{l+m+mi}{32}\PYG{p}{,} \PYG{l+s+s2}{\PYGZdq{}}\PYG{l+s+s2}{file.wav}\PYG{l+s+s2}{\PYGZdq{}}\PYG{p}{)}
\end{sphinxVerbatim}

\end{fulllineitems}



\chapter{Indices and tables}
\label{\detokenize{index:indices-and-tables}}\begin{itemize}
\item {} 
\sphinxAtStartPar
\DUrole{xref,std,std-ref}{genindex}

\item {} 
\sphinxAtStartPar
\DUrole{xref,std,std-ref}{modindex}

\item {} 
\sphinxAtStartPar
\DUrole{xref,std,std-ref}{search}

\end{itemize}


\renewcommand{\indexname}{Python Module Index}
\begin{sphinxtheindex}
\let\bigletter\sphinxstyleindexlettergroup
\bigletter{w}
\item\relax\sphinxstyleindexentry{wavpy}\sphinxstyleindexpageref{intro:\detokenize{module-wavpy}}
\end{sphinxtheindex}

\renewcommand{\indexname}{Index}
\printindex
\end{document}